\frontmatterpage
\unchapter{Abstract}

Prediction of student academic performance is a recently more often addressed research problem in higher education that can help universities tackle student dropout and retention problems in context of increased availability of digital educational data. Transparent and interpretable models predicting the expected student academic performance can be used for early identification of at-risk students and the implementation of evidence-based, academically-focused interventions. The Bachelor thesis develops and evaluates an information technology-based analytical solution for student academic performance prediction, using supervised machine learning methods and explainable artificial intelligence techniques on a recent publicly available student data.

The Bachelor Paper consists of three main chapters. Chapter 1 reviews the theoretical background of academic performance prediction, educational data mining, supervised machine learning methods, and explainable artificial intelligence, establishing the scientific foundation for the study. Chapter 2 presents the research methodology, including dataset selection, data preprocessing, feature engineering, and the overall experimental framework used to ensure methodological validity and reproducibility. Chapter 3 contains the practical research, where machine learning models are developed, evaluated, and interpreted using experimental results obtained from a real-world student performance dataset.

The Bachelor Paper has the methodological basis that is realized through the literature review of scientific publications in the field of student performance prediction and classification, the data preprocessing workflow, the supervised classification modelling with a real-world dataset, model performance evaluation with multiple classification metrics that are independent of the overall dataset label accuracy and post-hoc model explainability analysis. Logistic Regression, Random Forest and Gradient Boosting based models are implemented, used on a real-world student performance dataset and compared using accuracy-independent metrics (precision, recall, F1-score, ROC-AUC) and their explainability is analyzed using the applied explainability methods for the most important variables interpretation.

The experimental results show that the used ensemble-based machine learning models provide a strong predictive performance and the model analysis indicates that the progression variables, especially the curricular unit completion and the obtained grades, are the most indicative predictive features. The use of the applied explainability analysis methods on the developed predictive models confirms the overall model transparency and practical interpretability for the future use in early-warning and academic monitoring systems.
  
\thesisscopeEN

