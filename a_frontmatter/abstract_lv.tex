\frontmatterpage
\unchapter{Anotācija}

Studentu akadēmisko sasniegumu prognozēšana ir kļuvusi par nozīmīgu pētījumu virzienu augstākajā izglītībā, ņemot vērā studējošo atbiruma problēmas un digitālo mācību datu pieaugošo pieejamību. Precīzi un interpretējami prognozēšanas modeļi ļauj savlaicīgi identificēt studējošos ar paaugstinātu akadēmisko risku un atbalstīt datos balstītu lēmumu pieņemšanu izglītības iestādēs. Šis bakalaura darbs ir veltīts informācijas tehnoloģijās balstīta analītiska risinājuma izstrādei un novērtēšanai studentu akadēmisko rezultātu prognozēšanai, izmantojot uzraudzītās mašīnmācīšanās metodes un izskaidrojamas mākslīgā intelekta pieejas, kas pielietotas mūsdienu publiski pieejamiem studentu datiem.

Bakalaura darbs sastāv no trim galvenajām nodaļām. Pirmajā nodaļā tiek apskatīts studentu akadēmisko sasniegumu prognozēšanas teorētiskais pamats, izglītības datu ieguves metodes, uzraudzītās mašīnmācīšanās algoritmi un izskaidrojama mākslīgā intelekta koncepcijas. Otrajā nodaļā ir aprakstīta pētījuma metodoloģija, tostarp datu kopas izvēle, datu pirmapstrāde, pazīmju veidošana un eksperimentālā pētījuma struktūra. Trešajā nodaļā ir izklāstīta praktiskā daļa, kurā tiek izstrādāti, novērtēti un interpretēti mašīnmācīšanās modeļi, balstoties uz eksperimentāli iegūtiem rezultātiem no reālas studentu akadēmisko datu kopas.

Pētījuma metodoloģiskais pamats ietver zinātniskās literatūras analīzi, strukturētu datu apstrādi, uzraudzīto klasifikācijas modeļu izstrādi, modeļu novērtēšanu, izmantojot vairākus klasifikācijas rādītājus, kā arī post-hoc izskaidrojamības metožu pielietošanu. Praktiskie rezultāti apliecina, ka ansambļa tipa mašīnmācīšanās modeļi nodrošina augstu prognozēšanas precizitāti un ka studiju progresu raksturojošie rādītāji ir nozīmīgākie akadēmisko sasniegumu ietekmējošie faktori. Darba rezultāti pierāda piedāvātās pieejas piemērotību agrīnās brīdināšanas un akadēmiskās uzraudzības sistēmām.
  
\thesisscopeLV
